\documentclass[12pt]{article}
\usepackage{graphicx}

\oddsidemargin  -0.5 cm
\evensidemargin 0.0 cm
\textwidth      6.5in
\headheight     0.0in
\topmargin      -1 cm
\textheight=9in

\renewcommand{\arraystretch}{1.25}

\title{Alignment of the CMS Muon Endcap \\ with Tracks in Overlapping
  Pairs of Chambers \\ (CMS Note version)}
\author{Jim Pivarski, Alexei Safonov, Karoly Banicz}

\begin{document}
\maketitle

\begin{abstract}
We present a method for aligning the endcap muon chambers (and the
layers within them) using tracks that pass through pairs of
neighboring chambers in the same ring.  A muons's trajectory within an
endcap ring doesn't pass through any thick layers of iron, so the
trajectories of these tracks are nearly free of multiple scattering
effects and are therefore highly predictable.  Included in this Note
are results from Monte Carlo simulations and from real beam-halo data
from the September~2008 run of the LHC, in which we demonstrate a
270~$\mu$m $r\phi$-alignment of chambers using tracks from 9~minutes of
collected beam.  This is roughly the intrinsic resolution of the
chambers and within the scope of our ``long-term'' alignment goals.

\end{abstract}

\section{Motivation, Geometry, and Coordinate Systems}

The momentum resolution of any tracking system is determined by the
intrinsic hit resolution its subdetectors and one's knowledge of the
positions of those subdetectors in space.  Since these two sources of
uncertainty add in quadrature for any charged particle track, they
tend to be dominated by the worst of the two.  The 200--300~$\mu$m hit
resolution of the Cathode Strip Chambers (CSCs) in the CMS endcap muon
system has been well demonstrated by studies of the chambers in
isolation~\cite{intrinsic_resolution}.  Now that the whole CMS
detector has been assembled and is taking data, it is important to
learn the relative positions of all its elements (that is, ``align the
detector'') with the same degree of accuracy.

\subsection{Methods of Alignment and their Motivations}

There are several independent approaches to alignment, and they can be
used to verify one another.  The first and most straight-forward is
photogrammetry: hundreds of high-precision photographs were taken of
the assembled system and fed into a computer to fit for the positions
of all the chambers.  Two alignment pins are built into each CSC; they
are physically attached to all the layers that sense the passage of
charged particles and are visible from the outside.  Every pin was
capped with a reflective disk and illuminated with a bright light, so
that they glowed as easy-to-reconstruct circles in the photographs.
Though the disks are about a centimeter in radius, their centers were
reconstructed with a precision of about
300~$\mu$m~\cite{photogrammetry}.

Unfortunately, photogrammetry can only be performed while the muon
system is ``open,'' before the largest components are slid into place
(leaving no room for cameras) and the magnetic field is turned on.
The CMS magnetic field has a profound effect on the shape of the
detector, applying forces on the thick iron disks of the endcap which
are many times their weight, bending them by a centimeter in the
middle.  Under these forces, the individual chambers might shift
unpredictably, invalidating the photogrammetry result.  We must
therefore be able to align the system again while it is in operation.

To achieve this, the detector is also instrumented with
continuously-active alignment devices--- lasers, digital sensors,
inclinometers (like a carpenter's level), and mechanical calipers.
All of this information is read out in a separate stream from the
detector data and compiled into a global geometry description.  The
Muon Hardware Alignment System (MHAS) therefore has the following
advantages:
\begin{itemize}
\item it measures positions using physical instruments, as a ruler
  would,
\item it can observe the evolution of the system from the state
  measured by photogrammetry into a state where it can start taking
  data,
\item and it can measure any component of displacement, including
  those that are parallel to the trajectories of charged particle
  tracks.
\end{itemize}
However, the MHAS only monitors the positions of 6 out of the 18 to 36
chambers in each endcap ring, and requires great care to translate
device read-outs into global chamber positions.  To derive a geometry
which is useful for track reconstruction, the measurements must be
propagated in two directions: outward to all the other tracking systems,
through a diverse chain of devices, and inward from the devices
mounted on the chamber frames to the sensitive layers which actually
measure the passage of charged particles.  These issues are being
studied carefully, but for a complete understanding of the muon system
alignment, we will need another independent method.

A standard technique is to use the tracks measured by the detector
itself for alignment.  Muon tracks from LHC collisions traverse the
entire CMS detector in a continuous line, simplifying the alignment of
distant subdetectors, such as the muon system and the central silicon
tracker.  Also, the local hit measurements that make up a track are
automatically in the right coordinate systems for aligning the
sensitive components, rather than their physical structures.
Track-based alignment methods involve a cycle of fitting tracks with a
trial detector alignment, observing systematic offsets in the hit
residuals (difference between the best-fit track prediction and the
hit measurement on a given detector element), and using these to
correct the trial alignment.  Sometimes the cycle is an explicit
iterative loop, and sometimes it is a combined fit.

Track-based alignment also has its challenges, in that
track-propagation errors limit the precision and accuracy of the
detector alignment.  Multiple scattering in the iron return yoke of
the CMS solenoid introduces an independent error into each track,
broadening the residuals distributions, making it harder to see the
offset in the mean due to misalignment.  This type of uncertainty can
be cured with high statistics.  More serious are systematic
distortions to all tracks passing through a given region of the
detector, such as an error in the magnetic field description or a
misalignment in a part of the silicon tracker.  In these cases, one
might accidentally absorb non-misalignment errors into the detector
geometry, making it harder to solve these problems.  Fortunately, we
can also identify them using a number of track-based techniques,
including
\begin{itemize}
\item closed loops of relative alignment measurements ($B$ relative to
  $A$ and $C$ relative to $A$ compared to $B$ relative to $C$) using
  a broad distribution of track entrance angles (cosmic rays),
\item distinguishing magnetic field effects, which vary as a function
  of momentum ($\vec{B}\times\vec{p}$) and flip sign with charge, from
  misalignment effects which are independent of the set of tracks
  used in the study,
\item identifying the smoking gun of a real misalignment by observing
  the discontinuity in residuals at the known boundaries between detectors,
\item and averaging over localized errors by combining tracks from a
  large tracking volume in a single alignment measurement.
\end{itemize}
Most of these techniques are beyond the scope of this Note, and will
be described in a follow-up document on track-based muon alignment in
general.  They are part of an ongoing, integrated project of aligning
all the tracking systems and correcting errors in CMS's global track
model.

In this Note, we will present one completed part of the project, the
development of a procedure to align CSCs in the endcap rings using
tracks that pass through overlapping chambers.  This ``CSC Overlaps''
alignment procedure avoids all of the above issues because it depends
only on the parts of the tracks that are between thick layers of iron,
where multiple scattering is negligible and the magnetic field is
parallel to the beamline (because it jumps across the air gap through
the chambers, minimizing the lengths of its field lines).  This
procedure is both precise in a statistical sense because the minimal
multiple scattering leads to narrow residuals distributions, and is
proven to be accurate because it exactly reproduces the geometry known
from photogrammetry.  It demonstrates that we will be able to very
quickly align the endcaps with LHC beam-halo or collisions, in that
the so-called ``long-term'' (or ``100~pb$^{-1}$'') goal of
200--300~$\mu$m accuracy is achieved in 9~minutes of LHC beam from the
short September~2008 run.

\subsection{Geometry and Coordinates of the Muon Endcap System}





\section{Track-based Method in Detail}
\label{sec:details}

\section{Simulation Results}

\section{Real-Data Results}

\section{Preliminary Results for Layer Alignment}

\section{Concluding Remarks}



\end{document}
